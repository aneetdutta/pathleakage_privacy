%\clearpage
%\newpage
\section{Background}

Mobile phones have become commonplace and essential communications tools. They are used for phone calls, internet access, text messages, and documenting the world. Mobile phones' most profound privacy threat—yet often completely invisible—is how they announce your whereabouts all day (and all night) through the signals they transmit.  Location privacy leakage through communication protocol is an active area of research~\citneed. The user's location privacy is leaked by obtaining the identifier associated with the user and its location. 


\vspace{10pt}
\noindent
\textbf{Identification.}
Devices use identifiers to distinguish themselves from other devices. Static identifiers lead to attacks such as impersonation, privacy leakage, location tracking, identity mapping, remote deregistration attacks, etc. \cite{privacy1}. These attacks have motivated the use of randomized identifiers. 

\textit{Cellular (LTE/5G)  Tracking} is performed by capturing the International Mobile Subscriber Identity (IMSI) number that identifies a particular subscriber's SIM card. This identifier can be collected through an IMSI catcher  (fake base station)~\citneed  or stealthy attacks like signal overshadowing~\cite{ltrack}. The 3GPP proposed using Temporary Mobile Subscriber Identity (TMSI) instead of IMSI to prevent these attacks. These temporary identifiers are randomized at random time intervals, during handover, and \msnote{@Aneet in what other instances does TMSI change?}. TMSI is associated with C-RNTI, a MAC layer identifier. The C-RNTI value uniquely identifies the UE within a particular cell, which is essential for managing and delivering data. It is used in the uplink and downlink to multiplex and demultiplex the data and control information, as well as in the resource allocation. 


\textit{Bluetooth and WiFi} are short-range systems, compared to LTE/5G. The tracking through Bluetooth and WiFi is mostly associated with the leakage of the MAC layer identifiers. The Bluetooth Classic protocol has a static identifier known as Bluetooth address (BDADDR), a globally unique identifier for each device. The newer Bluetooth specification specifies that the Bluetooth Low Energy protocol should periodically randomize Bluetooth Device Address (BDA).

The WiFi devices...\msnote{@Aneet - complete this paragraph. Read the paragraphs above for the structure. }

\vspace{10pt}
\noindent
\textbf{Localization.} The localization accuracy of the device is increasing rapidly. The most influential driver for mobile location accuracy may be the regulatory requirements for mobile emergency calls (E911) set by the FCC, but a long list of new and envisioned human-to-machine and machine-to-machine use cases also influences it. These use cases include augmented reality, fitness wearables, real-time location-based advertising, transportation, package tracking, asset tracking, factory vehicles, shared bikes, and many more. As shown in Figure~\ref{fig:background_localization}, many use cases demand location accuracy in the meter and decimeter levels in indoor and outdoor scenarios. In order to meet these requirements, wireless protocols are evolving to provide new physical layers and distance measurement techniques. For example, WiFi systems have introduced Fine Timing Measurement (FTM) in IEEE 802.11mc,  which is further improved in Next Generation Positioning IEEE 802.11az. \msnote{Discuss later, what to keep? }  
\begin{figure}
    \centering
    \includegraphics[width=1\linewidth]{Example_scenario.png}
    \caption{\textcolor{red}{}{Screenshot, need to be replaced}}
    \label{fig:background_localization}
\end{figure}


Even when wireless systems are not designed particularly for localization, they are optimized to provide high Quality-of-Service (QoS). To ensure high QoS, the newer versions of WiFi (e.g., IEEE 802.11ad) and 5G enable wider bandwidth, beamforming, and mm-wave frequency range. While these advancements are necessary for these location-dependent applications and high QoS, they directly impact the location privacy of the user. By observing the physical layer characteristic of the signal (strength, phase, and frequency)~\citneed, or round-time-time instances~\cite{Domiens_FTM_privacy_paper}, an attacker can estimate the accurate position of the user.

LTE/5G systems use tight time synchronization between the base station and user equipment to prevent signal interference. By observing the arrival time of the signal originating from the base station and User Equipment (UE), an attacker can localize UE. The localization accuracy can be improved if these systems use higher bandwidth and beamforming~\citneed. While devices operating in GSM had localization errors of more than a kilometer, the LTE and 5G devices can be localized at the meter level accuracy~\citneed. Each wireless protocol has different physical layer properties; therefore, the localization accuracy varies for each of these systems. Nonetheless, the features of the physical layer that allow for better QoS  (such as wider bandwidth) directly impact the accuracy at which an eavesdropper can localize a device. 





% Wireless devices are being used increasingly in networking. While they tend to facilitate faster utility and growth, privacy of the devices must also be ensured. This leads to two major questions: 
% \begin{itemize}
%     \item How can we anonymize the devices and protect them from being identified?
%     \item How can we protect the devices from being tracked and located persistently?
% \end{itemize}

% \subsection{Device Identification}
% Devices identifiers are typically used by the devices to identify themselves from other devices. These identifiers could be string of numbers and letters generated uniquely. While the identifiers help to recognize the device, they could get compromised to an extent where the privacy of the user device could get violated. With static identifiers, various attacks like impersonation attack, privacy leakage attack, location tracking, identity mapping, remote de-registration attack, etc could be found \cite{privacy1}. These attack vectors have led to the evolution of randomization of the identifiers in order to break the identification linkability of the device. Identifier randomization is a process where the identifiers are created in temporal fashion for a particular time period after which they cease to exists. These temporary identifiers help to secure the identity of the device without providing any of its linkage to the eavesdropper. The eavesdropper would always tend to be fooled in recognising the device due to the randomization of identifiers. 

% This technique is commonly used in the wireless world. Various wireless technologies like bluetooth, wifi and cellular networks have been using identifier randomization in order to protect themselves from the privacy leakage. The static Bluetooth Device Address(BDA) which is referred as Bluetooth MAC Address is randomized in Bluetooth Low Energy (BLE) in the Bluetooth 4.0 specification. In Wifi, the device MAC Address is also randomized to increase the user privacy. In cellular networks, various identifiers are used at multiple layers of the protocol stack. Many of the temporary randomized identifiers are generated over the static identifiers to protect the static identifiers being disclosed. 






% In any wireless communication, quality of service(QoS) is an important performance metric that depends largely on the localization accuracy of the devices. Therefore, improving the localization accuracy for the devices is an active area of research for every communication protocol, and different techniques are proposed for that purpose. For example, in WiFi and 5G, beam-forming techniques are proposed for better localization. Also in 5G, MIMO architectures, millimeter waves and time difference of arrival(TDoA) techniques are proposed for more accurate localization. Bluetooth is a short-range communication, and distance measurement can be accurate from signal strength values. 

% The main advantage of better localization for wireless communications is that it allows one to understand the spatial distribution of the network. With clarity over spatial distribution, the base stations or the access points can allocate better resources (like allocating more resources to the regions with a high density of devices) and minimize interference in the signals, thus improving the QoS. Better localization will also help in the dynamic reuse of frequency bands, increasing the throughput of the devices. Therefore, improving the localization accuracy increases the throughput of the devices and hence improves the QoS. 

%  With better localization of devices, the risk of privacy violation. Since the devices can now be localized with more precision, the risk of privacy violation also increases. Since, now every device can be localized with more precision, every identifiers can now be associated uniquely with a device. Therefore, with precise location the identifier change can also be tracked. In order to prevent location tracking of the devices, each communication protocol uses temporary identifiers that are randomized after a particular time. For example, Bluetooth and WiFi use randomized MAC addresses, and LTE/5G uses temporary identifiers like C-RNTI and TMSI.




























% \section{Related Work}

% \subsection{Privacy in Wifi}

% \cite{wifi_randomization}




% \subsection{Privacy in LTE}


% The authors in \cite{6927731} mention that the IMSI could be linked with the C-RNTI within various different cells, but they fail to provide any substantial details related to the linkability or privacy leakage. Rather they choose to focus more on the countermeasures of possible privacy use-cases. In \cite{10.1145/3319535.3354263}, the authors provide a formal verification based framework that reveal various exploitable protocol design weaknesses. However, they do not mention anything related to the privacy leakage due to linkability. In \cite{breaking_layer_two}, the authors have successfully mapped the user's temporary network identifier (TMSI) to the radio identifier (RNTI). However, they have not provided any details for linkability between the C-RNTI and its scenarios that lead to privacy leakage. The authors in \cite{10.1145/2740070.2626320} provide the first ever prober "LTEye" that links the Physical layers Ids across the control messages for each user. The authors have tried to correlated the identifiers from the old logs with the identifiers to the new logs. However, they fail to provide any linkability in the identifiers and the scenarios that our papers bases upon. In \cite{277192}, the authors have provided more inputs related to IMSI-TMSI mapping and also have introduced the exposure of IMSI actively. In \cite{gutidemystified}, the authors suggest that the TMSI value in paging messages could lead to privacy leakage but do not provide any details regarding the other identifiers. In \cite{10526}, the authors provide more details related to the location leakage. In \cite{https://doi.org/10.1049/cmu2.12327}, the authors provide an overview of the passive attacks that could be performed included those related to C-RNTI. But they provide no reference to any real attack or any further details on linkability.


