\section{Attack Strategy}



\subsection{Device Model}

Every device currently has multiple communication protocols necessary for different use cases. 
Each communication protocol has different features that make it suitable for different applications.
Bluetooth, WiFi, and LTE/5G are the most common communication protocols deployed in a smartphone or smart device. In the following text, we will discuss the most commonly deployed protocols in details.


\textbf{Bluetooth}
Bluetooth is typically a short range communication with range between $10$ meters to $30$ meters. Nowadays, every smartphone device supports two variations of Bluetooth protocols, one is Bluetooth classic used for continuous data transfer at high rates and another is Bluetooth Low Energy(BLE) that is used for sending short burst of data at regular intervals commonly used for connecting to fitness trackers or other smart devices. BLE broadcasts the packets to all the devices in its proximity at regular intervals.

\textit{Bluetooth Temporal Identity} Bluetooth Classic protocol have a static identifier known as Bluetooth address(BDADDR) which is globally unique to identify a device. The identifier BDADDR does not gets randomized. But the frequency of transmission of Bluetooth Classic packets are less in number. Bluetooth Low Energy protocol by design is supposed to have randomized MAC address. Generally, the rate of randomization is periodic in nature which is in between $15$ to $45$ minutes depending upon vendor implementation.  The rate of transmission of packets is extremely high in BLE which is in order of hundreds/minute. 

\textit{Bluetooth Localization}
For device localization, physical layer properties like signal strength(RSSI) can be utilized. Since, Bluetooth is a short range communication, localization of devices using the signal strength at the receiver is quite accurate due to low signal interference.     High accuracy of localisation of  devices can be obtained by measuring the signal strength property at physical layer with an error of measurement in the order of $0.5$ to $1$ m level. 

 

\textbf{WiFi}

The typical range of communication in WiFi is between $50$ to $100$ meters depending upon the router. Each WiFi devices transmits WiFi probe requests at regular intervals containing the searched network names commonly termed as SSIDs. Each WiFi probe request is associated with an MAC address. Therefore, MAC address of each device can be easily obtained at regular intervals by any device with sniffing capability. Randomization of MAC address in WiFi can be of two types depending upon the vendor implementation.
\begin{itemize}
    \item \textit{Persistent MAC address Randomization.} In persistent MAC address randomization, the device does not communicates using its original MAC address which is static and globally unique in nature but it communicates with a randomly assigned MAC address. However, in the persistent MAC randomization scheme, the randomized MAC address depends upon the network profile and SSID of the WiFi access point. This MAC address does not get randomized further even if any device reconnects to the same WiFi access point. This scheme can be understood as anonymization of MAC address rather than randomization. Because, it only prevents from sending the original MAC address but it remains static once a random MAC address is assigned for a particular WiFi access point.
    \item \textit{Non-persistent MAC address Randomization.} In non-persistent MAC address scheme for WiFi, the MAC address randomization happens whenever a connection of a device is idle for a time in order of hours or when a device gets reconnected.
    
\end{itemize}

For localizing a device based on WiFi signal, physical layer signal properties like RSSI can be used which also provides us a localization with an error of $1$ to $3$ meters. However, in WiFi6 beam-forming techniques are proposed. With the beam-forming techniques proposed, the localization of device can be performed using Angle of Arrival(AoA) estimation, direction of arrival(DoA) estimation and fingerprinting of the beams at different locations. This will lead to an accuracy of device localization in order $cm$.


\textbf{LTE}

LTE is the defacto cellular communication technology deployed in every smartphones for constant connectivity. The communication range of LTE is large in comparison to the earlier communication protocols discussed. The communication range for LTE is about $3$ to $6$ kilometers. In LTE, the UE establishes connection with a base station for establishing the connection.  Since, LTE guarantees constant connectivity of the device, different procedures are invoked by the UE to maintain its connectivity to the base station considering mobility of an user. Handover is an important procedure in such a scenario where UE gets connected to one base station from another base station where the signal quality is better. Another important procedure that UE invokes in LTE when radio-link failure happens which is quite common due to obstacles and interference due to continuous signal transmission.

LTE protocol guarantees constant connectivity of the user and each base station can provide connectivity to a large number of devices. Therefore considering the scale of the protocol and different user scenario (due to mobility of the users), LTE protocol is complex and multiple identifiers are associated in the protocol.

Unlike, other communication protocols where an individual device  can be identified by MAC address, in LTE there are multiple identifiers associated at different layers of the protocol. Each UE have an unique identifier called IMSI which is globally unique. However, in order to prevent user tracking, a temporary identifier is now used known as TMSI which is randomized at different times and IMSI of a device is never broadcasted.

\textbf{LTE identifiers}. IMSI is a globally unique identifier of any user equipment. However, considering the privacy violation and user tracking, at present UE or base station does not transmits the IMSI and TMSI is used for communication instead. TMSI is a temporary identifier replacing IMSI which is an temporary identifier and gets randomized at different conditions and time intervals. C-RNTI is another important identifier at the MAC layer which is unique for every device connected to a particular base station. C-RNTI value is utilized by the base station for identification user device. In the below section we will discuss how this identifiers are used in particular scenarios.

\begin{enumerate}
    \item When a UE device connects to base station, it sends random access request(RAR), with a random C-RNTI value in the radio-link layer. Base station then assigns a particular C-RNTI value to the device in the random access response and is used for identification of the unique identification of the devices at the radio-link layer. Following, the RAR message exchanges, the UE sends RRC connection request using its TMSI identifier. And finally base station sends RRC connection setup complete packet containing both the C-RNTI value and TMSI value.
    \item In case of handover, RRC connection reconfiguration message is exchanged between the UE and the new base station using the C-RNTI value assigned by the earlier base station.
    \item In radio link failure scenario which is quite common in LTE, RRC reestablishment request is sent and when the connection is reestablished new C-RNTI value is assigned to the UE. 
\end{enumerate}



\textbf{LTE user localization.}

In LTE protocol, different users are connected to the base station from different locations resulting in different propagation delays. In order to adjust the delays, frame synchronization is necessary between the UE and the base station. Timing advanced commands are used for frame synchronization. Since, devices at greater distance will have more propagation delays, higher timing advanced commands are assigned to the that devices. Therefore, timing advanced values are used for localizing the device. The devices within $78$ meters assigned timing advanced value $1$ which is quite large in radius considering the accuracy of localization. However, by observing the timing advanced values and measuring the propagation delay of the signal the user can be localized with an error in the range $20$ meters.

In 5G protocols, MIMO architectures and beam-forming techniques are proposed which will increase the accuracy of device localization in $cm$ level.


\begin{table}[]
\centering
\resizebox{\columnwidth}{!}{%
\begin{tabular}{@{}|l|l|l|l|l|@{}}
\toprule
Protocols         & Identifiers  & Randomization & Localization Techniques  & Accuracy($\Delta D$) \\ \midrule
Bluetooth Classic & MAC address  & No            & RSSI                     & $\pm 1m$   \\ \midrule
BLE               & MAC address  & Yes           & RSSI                     & $\pm 1m$   \\ \midrule
WiFi              & MAC address  & Yes           & RSSI, beamforming        & $\pm 1m$  \\ \midrule
LTE               & TMSI, C-RNTI & Yes           & Timing Advanced commands & $\pm 78m$  \\ \bottomrule
\end{tabular}%
}
\end{table}
