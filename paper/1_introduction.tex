\section{Introduction}

%\subsection{Topic sentences}




Smart devices are central to the technology revolution witnessed in the last decade. With the advent of smart devices various services like health monitoring applications, banking, e-commerce, social media and many others are introduced. This has led to people wanting to stay connected by means of smartphones and other handheld electronic gadgets.

\begin{figure}[h]
  \centering
  \includegraphics[width=0.5\columnwidth]{images/communication_modules.png}
  \caption{Different communication modules of a device}
  \label{fig:comm_module}
\end{figure}


In order to maintain the constant connectivity smartphones use various communication protocols like LTE, WiFi and Bluetooth. Since, now people always carrying their smartphones and are constantly stay connected via the mentioned communication protocols, certain privacy risks like user tracking do exist.

User tracking is a huge privacy concern wherein a malicious party can find the location traces of individual users for a prolonged period of time. The location traces of user devices can reveal other sensitive information about the users such  as travel patterns, hobbies, workplace, presence or absence of an individual at a particular place at particular time and lot more. Therefore, it is important to prevent user tracking and collection of location traces continuously for serious privacy reasons.

In order to address the privacy risks and prevent user tracking, randomized temporary identifiers are proposed. The main idea behind using the temporary identifiers is that after regular intervals these identifiers will be changed and the attacker can not relate the newly changed identifier to the old identifier and thus prevent continuous tracking. For communication protocols like Wifi and Bluetooth MAC addresses are randomized. In LTE temporary identifier TMSI is now used instead of IMSI which used to be globally unique and static in nature.

However, in this work we try to address the question that whether randomizing the individual identifiers after certain interval of time is enough to protect the privacy of the users? Since, the communication protocols mentioned above work independently in a device, a pertinent question to ask is, does the independent randomization of the individual identifiers separately ensures privacy preservation? Or is it necessary to analyze the randomization technique for all the communication protocols together to analyze the privacy of the device as a whole?

Any user uses all the communication protocols simultaneously for different services. For internet surfing, user uses WiFi, for connecting with Bluetooth headphones or fitbit devices while traveling it uses Bluetooth and for telecommunication services it uses LTE. Even if the user is not utilizing all the services at once, but the users are reluctant to switch-off these communication protocols when not in use. For example, most of the people nowadays connects to WiFi services in home or in office. But when they leave home or office they do not turn off the WiFi protocols in their devices so that it can automatically gets connected to home WiFi or office WiFi when they reach their destination. Therefore, the privacy of any user should be evaluated by based on considering all the communication protocols together and not by evaluating the effectiveness of the privacy of individual communication protocols.

The existing line of work concerned about the privacy of users related to identifier randomization in communication protocols deals with individual protocols. In \cite{ble_phy} the technique for fingerprinting of user devices based on Bluetooth advertisement signals are proposed. The attackers will be able to track the fingerprinted devices based on the BLE advertisement packets sent at regular intervals. In \cite{ble_linking} privacy issues in Bluetooth is again addressed. In this paper, Bluetooth Classic and Bluetooth Low Energy advertisement signals are correlated based on the timing intervals of the advertisement broadcast. This correlation is useful for tracking since, Bluetooth Classic have an unique global identifier and BLE have a randomized identifier, therefore correlating the temporary identifier with globally unique static identifier will help in tracking the user. In \cite{tracking_ble} a static identifier is extracted from the information broadcast-ed from the BLE packets. The static identifier is then associated with the randomized BLE MAC identifier for tracking. The privacy analysis of WiFi MAC address randomization is also an well explored area. In \cite{wifi_randomization} the vulnerability in the randomization process of the MAC address is discussed with predictable scrambler seed. Also, in \cite{wifi_probe} shows how information elements in wifi requests can be used as an signature to uniquely identify a device. The signature extracted can be used as a fingerprint for tracking, failing the MAC address randomization. In LTE, \cite{lttrack} focused on LTE communication protocol for user tracking by extracking the IMSI by injecting some signal to the user equipment to force them send the \textit{attach} request. The IMSI extracted which is static and globally unique is binded with the temporary identifier TMSI for continuous tracking.

Considering the existing body of work focusing on privacy issues of individual communication protocols, to the best of our knowledge this is the first work that addresses the privacy concerns of the device level as a whole by considering the above mentioned communication protocols together. The main insight of this work is \textit{cross-linking} of different identifiers corresponding to various communication protocols to enable user tracking. In a given device since each communication protocol functions independently, identifier randomization happens at different time instance for different protocols. Therefore, the newly generated identifier of a given protocol can be linked to its former identifier by leveraging the unchanged identifiers corresponding to the remaining protocols. In this work we propose a new user tracking technique based on user localization technique that uses physical signal properties like signal strength (RSSI).  This combination of user localization and \textit{cross-linking} enables user tracking by the  adversaries for longer duration.

\textbf{\textit{Contributions.}}
\begin{enumerate}
    \item We demonstrate the feasibility of tracking devices without any need to extract globally unique static identifiers. The privacy analysis and user tracking mechanism proposed in this work is completely based on randomized temporary identifiers. 
    \item We analyze the cross-linking capability between the different randomized identifiers based associated with different communication protocols. The user tracking mechanism conducted in this work considers the functionality of all the communication protocols simultaneously to analyze the privacy of any device as a whole.
    \item We demonstrate a complete passive user tracking mechanism without any 
     signal injection or trigger is sent by the adversary to extract any information from the user.
    \item Particularly in LTE we found there is a possibility of cross-layer linking of two different identifiers namely TMSI and C-RNTI. Different scenarios are discussed in this paper while evaluating the tracking of any user based on LTE communication protocols.
    \item We proposed a formalized approach of quantifying the location privacy of any device based on the communication protocols used. This helps us to understand the root causes for privacy leakage and what are the necessary steps required for privacy preservation can be inferred.
   % \item We discussed some countermeasures
    
\end{enumerate}


%\textbf{Ethical Considerations.} In this work we intend to inform that even if randomization technique for identifiers for each communication protocols is 



 