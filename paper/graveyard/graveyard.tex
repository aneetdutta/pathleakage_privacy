

\section{Old Stuff from Threat Model}

In the context of user tracking and the motivation described in the case study above, we discuss about the threat model on this section.  The threat model considered for this work focuses on stealth nature of the attacker which does not send any active packets or signals from which any intrusion detection system or the user can know about the presence of the ongoing user tracking. The main reason for considering an attacker model with complete stealth mode because we want emphasize on the fundamental vulnerabilities of the communication protocols which results in leaking information enabling user tracking. 


In our considered threat model, the attacker can listen or sniff to any packets that is being transmitted between the base stations and user equipment(UE). The sniffer can read and extract information that is being publicly transmitted by the access points and UEs without breaking any encryption scheme. Generally, the identifiers in MAC layer for WiFi and Bluetooth is being transmitted in plain-text and can be easily be read by the sniffer knowing the presence of any devices within the sniffer's proximity. Similarly in LTE, MAC layer and radio link layer identifiers are obtained by the attacker in plain-text. Along with the identifiers, the attacker can read or measure physical layer signal properties like signal strength values, time of arrival of the signals, angle of arrival of the signals. Measuring the signal properties will help the attacker to localize the device resulting in continuous tracking of the user device.


Therefore, in the proposed we assume the following attacker's capability in our threat model:
\begin{itemize}
    \item The attacker can receive and read any information being transmitted in plain-text from the user and access points or base stations.
    \item The attacker have the freedom of placing the sniffer in any location.
\end{itemize}

Considering our threat model we argue that how the existing measures in communication protocols are not privacy-preserving and the attacker by simply receiving the packets transmitted publicly can enable user tracking. Specially attacker at state-sponsored level and high budget of placing as many sniffer they want can easily track the users failing the measures taken for privacy preservation at the current state.

