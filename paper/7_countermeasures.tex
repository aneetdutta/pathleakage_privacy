\section{Countermeasure}


The main objective of designing the countermeasure for such a privacy leakage attack is to minimize the duration of tracking by the adversary. Since, at the current state of the start scenario the cross linking between identifiers for different protocols is possible, any randomized identifier can be mapped back to its previous identifier failing the process of randomization. The countermeasure focuses on mitigating the correlation among the identifiers for different protocols such that the newly changed identifiers can not be mapped back to its previous identifier. The tracking by the adversary will only be limited to the time period for which that particular identifier is active. When the identifier gets randomized, the adversary can not be certain that if the newly changed identifier is a new device or it is a randomized identifier. The countermeasure technique prevents mass user tracking by introducing more uncertainty for the adversary to predict if the new identifier belongs to a new device or it is a randomized identifier from an existing device.

According to definition \ref{def:privacy_loc} the privacy of any device will increase when  the localization error of the device($\Delta D)$ is decreased, or the number of devices present within the localization error is increased, or the timing interval of randomizing the identifiers decreases. The challenges associated for proposing such a countermeasure technique is following.

\begin{enumerate}
    \item Since localization error is directly proportional to the quality of service(QoS), $\Delta D$ can not be made large in order to increase privacy. Moreover, with the advantage of localization for increase in QoS, $\Delta D$ will decrease over time especially with coming 5G technology using beam-forming techniques and MIMO architecture.
    \item $\Delta T$ at present is always high considering the temporal correlation between the identifiers possible due to randomization of different wireless communication modules happening at different times.  
    \item The factor of $\Delta n$ is not in direct control of the user since it can not predict how many devices will be in the vicinity of it to ensure privacy of the device. Also, with decrease in $\Delta D$, $\Delta n$ will also decrease.
\end{enumerate}

Therefore, considering the above factors for privacy leakage, we propose a technique to ensure privacy of the devices and prevent tracking for longer duration.

\subsection{Location aware identifier randomization(LAIR)}

The problem of linkability arise due to temporal correlation between the identifiers randomizing at different times, therefore if every identifier of a device gets randomized at the same time then the problem of temporal correlation can be mitigated. Therefore, in our mitigation technique, the identifiers associated with every protocol is randomized at the same time resulting in temporal correlation among the identifiers to fail. However, if in scenario where $\Delta n$ is low then simply randomizing all the identifiers for any particular device will be not enough for privacy preservation since the newly randomized identifiers can easily be associated with the previous identifier which does not exist any more. Therefore, the device should randomize their identities considering both the factors of temporal correlation($\Delta T$) and the factor of $\Delta n$.   

The proposed countermeasure technique for privacy preservation in wireless communication technologies works similar to mix zones as proposed in different routing applications to ensure privacy where the users in particular area randomizes their identities within the mix zone such that the adversary can not distinguish between the users going in to the mix zone with the users coming out of the mix zone. The proposed LAIR works as follows.
\begin{enumerate}
    \item Every device checks the number of device in the proximity. When $\Delta n$ is high, every device is notified to randomize their identifier for all the communication protocols, creating mix zone at any random location at random time.
    \item When $\Delta n$ is high, every device within the radius will change the identifier of all the communication protocols at once. The identifier associated with LTE is randomized by terminating the existing connection by sending RRC release request followed by RRC connection setup such that the cross-layer linkability does not exist within LTE protocol.
\end{enumerate}

\subsection{Privacy Analysis of LAIR}

LAIR provides privacy by considering two important factors in privacy leakage, i.e., the density of devices within the localization error radius($\Delta n$) and the timing interval of the identifier randomization($\Delta T$). For the following reason, randomizing all the identifiers for all the $\Delta n$ at the same time is necessary.
\begin{itemize}
    \item All the identifiers for a device are changed at once, therefore $\Delta T$ is the time interval for which the particular set of identifiers are active. The process of randomizing the identifiers for all the  communication protocols is similar to each device having a single identifier. Each time a new identifier is recorded by the adversary, it can either be a new device or any identifier getting randomized. Therefore, mitigating the temporal correlation among different identifiers introduces uncertainty to the adversary for successful tracking.
    \item If any particular device randomizes all their identifiers in a pool of device($\Delta n$), and other devices in proximity to that randomized device remains static then it can easily be inferred that the randomized identifier belongs the device identifier which does not exist anymore. Therefore, it is necessary to randomize multiple devices at once to increase the uncertainty for the adversary.
\end{itemize}

Since LAIR randomizes all the identifiers of a device at the same time. Therefore LAIR guarantees $\Delta T$ to be the time interval between two randomized set of identifiers. At the current state where temporal correlation between the identifiers is possible, $\Delta T$ tends to infinity($\infty$). Therefore, from definition \ref{def:privacy_loc}, LAIR increases the privacy of the device since it decreases the value of $\Delta T$. In a way, the user-tracking capability of the adversary is limited to the time duration of $\Delta T$.

In LAIR, when a device randomizes their identifiers then all the other devices in its proximity also randomizes their identifiers. Therefore, for each WiFi identifier there will be $\Delta n$ number of candidates of LTE identifiers. Due to cross-layer linking in LTE, $\Delta n$ will decrease over time. However, if $\Delta T$ can be set as such that $\Delta n$ does not become too low at any time then the privacy of the device can be preserved because, it will decrease the granularity of the location data for any particular device and also temporal correlation can not be executed by the adversary. To explain it clearly, let us denote $\Delta n$ at time instance $t_1$ as $\Delta n_{t_1}$ and $\Delta n$ at time instance $t_2$ as $\Delta n_{t_2}$ where $t_2>t_1$ and $\Delta n_{t_1} - \Delta n_{t_2}<\epsilon$. Here, $\epsilon$ is the minimum anonymity set required for privacy preservation. Let us assume $t_2$  is the first time instance where the condition $\Delta n_{t_1} - \Delta n_{t_2}<\epsilon$ is satisfied.
\begin{definition}
The necessary condition for privacy preservation against continuous user-tracking is such that the time interval between randomization of the identifiers is such that the size of the anonymity set is always greater than $\epsilon$
\begin{equation*}
    \Delta T < t_2 - t_1
\end{equation*}\label{def:lair}
\end{definition}

\textbf{Limitation.} The limitation of LAIR is that to maintain an anonymity set of size $\epsilon$, certain devices are always required within the proximity of the devices. Since, from definition \ref{def:privacy_loc} we can understand that the privacy of device is directly dependent upon $\Delta n$, therefore, when $\Delta n$ is very low, user tracking will be possible.

The randomization of identifiers based on certain conditions will lead to a performance issue due to the necessity of reestablishing the connection by terminating the existing connection with the WiFi access point or base station. 

