\section{Related Work}

 \subsection{Privacy in LTE}

\cite{breaking_layer_two} is the first paper in LTE to demonstrate the mapping between two different identifiers from two different layers namely TMSI and C-RNTI for a device. The paper discussed about the identity mapping and how it can violate privacy. However, only mapping the TMSI with C-RNTI during RRC connection request phase is not sufficient for continuous user tracking because the value of C-RNTI and TMSI changes at a later duration under different scenarios like radio link failure and handover where the adversary can only observe the changed C-RNTI value. Therefore, in this work we have demonstrated how an adversary can perform the entire tracking operation with the linkability within the C-RNTI identifiers and with cross-layer linking between the TMSI and C-RNTI at different scenarios especially handover.  

Jover et al. \cite{jover2016lte} demonstrates the possibility of user tracking with only C-RNTI identifier. The paper discusses how the changed C-RNTI during handover process can be used for tracking along with the cell ID associated with the device.  However, user tracking based on C-RNTI and cell ID have its limitation and also, within a particular cell the C-RNTI can get randomized under different conditions and also during certain scenarios during handover the C-RNTI of the device remains same but the TMSI gets randomized. Therefore, tracking any particular device with C-RNTI will result in adversary to loose the track of the device in much shorter duration if it is not mapped with the IMSI or TMSI value.


 Ghafghazi et al. \cite{privacy_public_safety} discusses about the location privacy in LTE by describing how IMSI identifier unique for every device can be associated with the C-RNTI within different cells of LTE communication protocol. However, the mentioned work does not focuses on user-tracking but it tries to mitigate the problem of privacy leakage by anonyomizing the IMSI identifier such that IMSI identifier is not sent by any device in plain-text and the IMSI value can be derived by the base station from a random string. In our work, we focus on user-tracking with only the temporary identifiers without any need for extracting any globally unique identifier.

In \cite{lttrack} authors have demonstrated how timing advanced value can be used for localization of devices with more precision, rather than just observing the cell ID of the base station. This work helps to localize devices with better accuracy in LTE protocol. However, for tracking the proposed method in the paper extracts the IMSI value from the attach request between the device and the base station. However, the attach request is being only sent by the device before establishing the connection with the base station and does not reveal the IMSI identifier after that. For continuous tracking, the adversary needs to perform signal overshadowing in a much more efficient way proposed in \cite{adaptover} to detach the device from the base station. Once the device is detached it resend the attach request and the adversary can monitor the IMSI with its precise location derived from timing advanced command. Since, the adversary needs to perform overshadowing to gain the IMSI of the device the privacy leakage attack is not entirely stealthy in nature. 

In \cite{shaik_2016} the authors have demonstrated user tracking with the granularity level of the cell ID associated with each base station. The tracking mechanism proposed is based on IMSI and the adversary needs to setup fake/rogue base station in order to identify every devices based on their IMSI and hence the privacy leakage attacks can be detected. In the passive version of the attack proposed in this paper the adversary observes the paging messages and maps the C-RNTI values with the respective IMSI values to get a coarse location of the UE.

\cite{gutidemystified} can track users location based on the GUTI identifier of the devices. The paper highlights the vulnerabilities in the GUTI reallocation strategy where the operator reallocates the same GUTI in a particular region.

\subsection{Privacy in WiFi}

\cite{wifi_mac} demonstrates how the user tracking can be possible by fingerprinting the information element(IE) of probe requests. The IE are used for fingerprinting the devices and can be used for identifying the devices and tracking. The paper also highlights the vulnerability in the MAC address randomization scheme of WiFi because of the predictable scrambler seed which allows linking of randomized MAC addresses of same device.

Matte et al. \cite{wifi_timing} demonstrate how the SSIDs transmitted during probe request can be used as an signature to identify the MAC addresses associated with any device. Overtime the SSID stored in the device remains same and the signature can be used as an pseudo-identifier for tracking. Also, in the paper the authors have demonstrated how the probe requests are sent in bursts within an time interval of $10ms$. Therefore, the probe requests from a single device can be identified based on the timing of the SSIDs and the signature can be used for tracking. However, in the current state the WiFi access points which are password protected, SSIDs are not broadcasted in plain-text. But, the SSIDs of the public WiFi access points are still visible and can be used as an signature.

\cite{asimov} presents a technique to continuously track a WiFi device based on device localization using RSSI and creating a signature based on information element of the WiFi probe requests.



\subsection{Privacy in Bluetooth}

\cite{blc_linking} links Bluetooth classic address(BDADDR) which is static in nature with the BLE address which gets randomized. Linking the Bluetooth classic and the Bluetooth low energy protocol enables tracking the BLE randomized identifiers with the help of static BDADDR. The vulnerability of linking the Bluetooth classic and Bluetooth low energy lies in the fact that both BLC and BLE protocols are implemented in the same chip such that the adversary can link the protocols based on the timing of the advertisement packets.

\cite{ble_phy} proposed a technique to identify BLE devices based on fingerprinting of BLE signals.

\cite{tracking_ble} performs tracking the anonymized Bluetooth devices by extracting a static identifier from the received BLE packets. For tracking, the authors proposed a carry-forward algorithm which can map back the randomized identifier to the old one with the help of the extracted static identifier.

However, the mentioned works considers tracking based in a single protocol which does not provide the measure of the privacy of the device as a whole. Therefore, in our work we highlight that even if all the randomization of all the protocols works perfectly fine then also lies the deeper vulnerability of cross-linking between the identifiers of different protocols for successful tracking by the adversary.












 




% \section{User Tracking for Surveillance}

% In this section, we discuss the method for physical surveillance of users based on cross-linking between different communication protocols like Wifi, Bluetooth and LTE. The main objective of the attacker is to track every users location and follow its trajectory for longer duration of time. The main sight of this paper is that, since a device like smartphone uses multiple communication protocols and each protocols deploys their randomization techniques of identifiers separately at different times, we can cross-link these identifiers to map the new identifiers to the old identifiers.

% \subsection{Case Study}

% The attacker aims to continuously track the locations of different users for mass surveillance to know the exact path trajectory of every user. Since, the location traces reveal a lot about users profession, habits, travel patterns and more, the location traces are considered as sensitive information. If the attacker is able to continuously identify a device and extract its location traces, it can reveal a lot more information and there is a serious breach of privacy. The attacker will be able to infer a lot of other private information about the users. Assuming the attacker can install sniffers in multiple locations in a city, it can continuously sniff the packets transmitted from the devices and can link them with previous packet identifiers sniffed at different locations inferring the entire location trajectories.

% The attacker considered in this work, can be state-sponsored attacker who have the capability to install multiple sniffers in multiple locations sniffing the data continuously. The attacker can also be a malicious company whose devices are installed all over the city which can be turned in to sniffers, for example traffic lights and digital hoardings. Also, the attack can be performed by individual attackers but the area of coverage might be less concerning the resource requirement simultaneously at the same time at different locations.

% \subsubsection{\textbf{Challenges}:} 

% For achieving continuous location tracking surveillance, there are two main challenges for the attacker. First, the communication protocol randomizes their temporary identifiers and it changes at random times. Therefore, the attacker needs to maps the old identifiers with the new identifiers for continuous tracking. Secondly, the attacker needs to localize the devices to associate the identifiers sniffed with the device and its corresponding location at a particular time.
