% \subsection*{Abstract}

With the advancement of smart devices, we constantly need to stay connected for different services, from navigation to social media to health monitoring applications. Communication protocol like WiFi, LTE, and Bluetooth is essential for any device to maintain its constant state of connectivity. However, constantly staying connected also increases the risk of privacy leakage regarding continuous surveillance. The earlier works highlight the vulnerability of the randomization schemes of the temporary identifiers of the particular protocols and analyze the privacy of implementation of the particular protocols. In this work, we highlight that even if the implementation of the randomization schemes for the temporary identifiers of a particular protocol is privacy-preserving, since the randomization of the identifiers for different protocols happens at different times, \textit{cross linking} between the identifiers is possible between the new identifier and the old identifier of some other protocol enabling the attacker to continuously track any devices by using the other unchanged identifier as a reference point. We analyze the privacy of any device as a whole and establish how it is related to user behavior of enabling and disabling the communication protocols of the device. In this paper, we demonstrate how the information broadcast-ed by different protocols at regular intervals, like WiFi, Bluetooth, and LTE, can be utilized for co-relating the signals sent by the same devices for different protocols. Once the co-relation is established, the attacker can continuously sniff the packets to perform the cross-linking and track the devices for a longer duration, even after the devices perform randomization. Additionally, in this paper, we have discussed the practical implementations and the default settings provided by the vendors for the randomization schemes of the identifiers. 
