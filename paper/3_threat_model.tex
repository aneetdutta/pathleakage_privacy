\section{Threat Model \& Attacker Goal}

We are concerned with an attacker that wants to track users' devices. In this paper, we assume that the adversary is passive. It eavesdrops on wireless communication between user equipment (UE) and base stations and other devices. From these received signals it tries to recover location traces of a device (and thus of a user). The attacker does not inject or modify any traffic. 

% MULTIPLE PROTOCOLS / SNIFFERS
We assume that the attacker has access to one or more sniffers, placed at different locations within the area of interest. We assume that these sniffers can receive multiple communication protocols. The sniffers for different protocols do not need to be colocated. For example, our attacker might have one LTE sniffer, covering a large area, and in addition several combined WiFi and Bluetooth sniffers covering specific locations.

% ATTACKER CAPABILITY: LOCATING RECEIVED SIGNALS
We assume that sniffers can locate the source of received signals to within a well-defined area. The size of this area can depend on the protocol, the true location of the device, and capabilities of the sniffer. The sniffer can use for example physical layer signal properties like signal strength values, time of arrival of the signals, angle of arrival, or triangulation to estimate a device's location.

% ATTACKER GOAL
The goal of the attacker is to recover location traces of devices, and thereby violate the privacy of the people carrying these devices. Since the eavesdropper can observe the temporary identifiers for each protocols, it can reconstruct partial location traces for each device-protocol pair -- it simply records all locations of each fixed identifier. In theory, however, the eavesdropper should not be able to stitch together these partial traces into longer device specific traces. Our attacker specifically tries to recover location traces that are longer than the time between identifier rotation of the protocols considered.

% OUT OF SCOPE: LINKING LOCATION TRACES TO PEOPLE
Following existing work, we focus on recovering location traces for devices, and do not aim to link these traces to individuals. It is well known that given location traces it is relatively easy to link these traces to individuals, for example by using specific points of interest such as home or work addresses, or by identifying targets at one point along the trace.

% DEFINITION OF PRIVACY
\noindent
\textbf{Effectiveness of the Attack.}
To capture the effectiveness of the attacker, we measure the maximum duration during which an attacker can recover a location trace of a specific device. To obtain a privacy metric, we normalize by the maximum duration that the attacker can observe the device.

\begin{definition}
  We say that the location privacy $\textsf{LP}_{\mathcal{D}}$ of a device $\mathcal{D}$ that is being observed by an attacker $\mathcal{A}$ (observations can come from different sniffers) recovering partial traces $\{ \textsf{trace}_i \}_i$ is defined as:
  \begin{equation*}
    \textsf{LP}_{\mathcal{D}} = \frac{\textrm{Length of longest $\textsf{trace}_i$ matching device $\mathcal{D}$}}{\textrm{duration during which device $\mathcal{D}$ was observed by $\mathcal{A}$}}
  \end{equation*}
\end{definition}
\wl{This is a bit of a rough definition. Feedback welcome}
% Should we be explicit about the observers?
The above definition does not require that the attacker is able to attribute a specific trace to a target device. This is deliberate. Existing techniques help with assigning traces to devices. We want to capture how well the attacker is able to recover traces in the first place.
\wl{Also: didn't link this yet to existing privacy definitions}
